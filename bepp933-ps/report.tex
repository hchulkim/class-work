% Options for packages loaded elsewhere
\PassOptionsToPackage{unicode}{hyperref}
\PassOptionsToPackage{hyphens}{url}
\PassOptionsToPackage{dvipsnames,svgnames,x11names}{xcolor}
%
\documentclass[
  letterpaper,
  DIV=11,
  numbers=noendperiod]{scrartcl}

\usepackage{amsmath,amssymb}
\usepackage{iftex}
\ifPDFTeX
  \usepackage[T1]{fontenc}
  \usepackage[utf8]{inputenc}
  \usepackage{textcomp} % provide euro and other symbols
\else % if luatex or xetex
  \usepackage{unicode-math}
  \defaultfontfeatures{Scale=MatchLowercase}
  \defaultfontfeatures[\rmfamily]{Ligatures=TeX,Scale=1}
\fi
\usepackage{lmodern}
\ifPDFTeX\else  
    % xetex/luatex font selection
\fi
% Use upquote if available, for straight quotes in verbatim environments
\IfFileExists{upquote.sty}{\usepackage{upquote}}{}
\IfFileExists{microtype.sty}{% use microtype if available
  \usepackage[]{microtype}
  \UseMicrotypeSet[protrusion]{basicmath} % disable protrusion for tt fonts
}{}
\makeatletter
\@ifundefined{KOMAClassName}{% if non-KOMA class
  \IfFileExists{parskip.sty}{%
    \usepackage{parskip}
  }{% else
    \setlength{\parindent}{0pt}
    \setlength{\parskip}{6pt plus 2pt minus 1pt}}
}{% if KOMA class
  \KOMAoptions{parskip=half}}
\makeatother
\usepackage{xcolor}
\setlength{\emergencystretch}{3em} % prevent overfull lines
\setcounter{secnumdepth}{-\maxdimen} % remove section numbering
% Make \paragraph and \subparagraph free-standing
\makeatletter
\ifx\paragraph\undefined\else
  \let\oldparagraph\paragraph
  \renewcommand{\paragraph}{
    \@ifstar
      \xxxParagraphStar
      \xxxParagraphNoStar
  }
  \newcommand{\xxxParagraphStar}[1]{\oldparagraph*{#1}\mbox{}}
  \newcommand{\xxxParagraphNoStar}[1]{\oldparagraph{#1}\mbox{}}
\fi
\ifx\subparagraph\undefined\else
  \let\oldsubparagraph\subparagraph
  \renewcommand{\subparagraph}{
    \@ifstar
      \xxxSubParagraphStar
      \xxxSubParagraphNoStar
  }
  \newcommand{\xxxSubParagraphStar}[1]{\oldsubparagraph*{#1}\mbox{}}
  \newcommand{\xxxSubParagraphNoStar}[1]{\oldsubparagraph{#1}\mbox{}}
\fi
\makeatother

\usepackage{color}
\usepackage{fancyvrb}
\newcommand{\VerbBar}{|}
\newcommand{\VERB}{\Verb[commandchars=\\\{\}]}
\DefineVerbatimEnvironment{Highlighting}{Verbatim}{commandchars=\\\{\}}
% Add ',fontsize=\small' for more characters per line
\usepackage{framed}
\definecolor{shadecolor}{RGB}{241,243,245}
\newenvironment{Shaded}{\begin{snugshade}}{\end{snugshade}}
\newcommand{\AlertTok}[1]{\textcolor[rgb]{0.68,0.00,0.00}{#1}}
\newcommand{\AnnotationTok}[1]{\textcolor[rgb]{0.37,0.37,0.37}{#1}}
\newcommand{\AttributeTok}[1]{\textcolor[rgb]{0.40,0.45,0.13}{#1}}
\newcommand{\BaseNTok}[1]{\textcolor[rgb]{0.68,0.00,0.00}{#1}}
\newcommand{\BuiltInTok}[1]{\textcolor[rgb]{0.00,0.23,0.31}{#1}}
\newcommand{\CharTok}[1]{\textcolor[rgb]{0.13,0.47,0.30}{#1}}
\newcommand{\CommentTok}[1]{\textcolor[rgb]{0.37,0.37,0.37}{#1}}
\newcommand{\CommentVarTok}[1]{\textcolor[rgb]{0.37,0.37,0.37}{\textit{#1}}}
\newcommand{\ConstantTok}[1]{\textcolor[rgb]{0.56,0.35,0.01}{#1}}
\newcommand{\ControlFlowTok}[1]{\textcolor[rgb]{0.00,0.23,0.31}{\textbf{#1}}}
\newcommand{\DataTypeTok}[1]{\textcolor[rgb]{0.68,0.00,0.00}{#1}}
\newcommand{\DecValTok}[1]{\textcolor[rgb]{0.68,0.00,0.00}{#1}}
\newcommand{\DocumentationTok}[1]{\textcolor[rgb]{0.37,0.37,0.37}{\textit{#1}}}
\newcommand{\ErrorTok}[1]{\textcolor[rgb]{0.68,0.00,0.00}{#1}}
\newcommand{\ExtensionTok}[1]{\textcolor[rgb]{0.00,0.23,0.31}{#1}}
\newcommand{\FloatTok}[1]{\textcolor[rgb]{0.68,0.00,0.00}{#1}}
\newcommand{\FunctionTok}[1]{\textcolor[rgb]{0.28,0.35,0.67}{#1}}
\newcommand{\ImportTok}[1]{\textcolor[rgb]{0.00,0.46,0.62}{#1}}
\newcommand{\InformationTok}[1]{\textcolor[rgb]{0.37,0.37,0.37}{#1}}
\newcommand{\KeywordTok}[1]{\textcolor[rgb]{0.00,0.23,0.31}{\textbf{#1}}}
\newcommand{\NormalTok}[1]{\textcolor[rgb]{0.00,0.23,0.31}{#1}}
\newcommand{\OperatorTok}[1]{\textcolor[rgb]{0.37,0.37,0.37}{#1}}
\newcommand{\OtherTok}[1]{\textcolor[rgb]{0.00,0.23,0.31}{#1}}
\newcommand{\PreprocessorTok}[1]{\textcolor[rgb]{0.68,0.00,0.00}{#1}}
\newcommand{\RegionMarkerTok}[1]{\textcolor[rgb]{0.00,0.23,0.31}{#1}}
\newcommand{\SpecialCharTok}[1]{\textcolor[rgb]{0.37,0.37,0.37}{#1}}
\newcommand{\SpecialStringTok}[1]{\textcolor[rgb]{0.13,0.47,0.30}{#1}}
\newcommand{\StringTok}[1]{\textcolor[rgb]{0.13,0.47,0.30}{#1}}
\newcommand{\VariableTok}[1]{\textcolor[rgb]{0.07,0.07,0.07}{#1}}
\newcommand{\VerbatimStringTok}[1]{\textcolor[rgb]{0.13,0.47,0.30}{#1}}
\newcommand{\WarningTok}[1]{\textcolor[rgb]{0.37,0.37,0.37}{\textit{#1}}}

\providecommand{\tightlist}{%
  \setlength{\itemsep}{0pt}\setlength{\parskip}{0pt}}\usepackage{longtable,booktabs,array}
\usepackage{calc} % for calculating minipage widths
% Correct order of tables after \paragraph or \subparagraph
\usepackage{etoolbox}
\makeatletter
\patchcmd\longtable{\par}{\if@noskipsec\mbox{}\fi\par}{}{}
\makeatother
% Allow footnotes in longtable head/foot
\IfFileExists{footnotehyper.sty}{\usepackage{footnotehyper}}{\usepackage{footnote}}
\makesavenoteenv{longtable}
\usepackage{graphicx}
\makeatletter
\newsavebox\pandoc@box
\newcommand*\pandocbounded[1]{% scales image to fit in text height/width
  \sbox\pandoc@box{#1}%
  \Gscale@div\@tempa{\textheight}{\dimexpr\ht\pandoc@box+\dp\pandoc@box\relax}%
  \Gscale@div\@tempb{\linewidth}{\wd\pandoc@box}%
  \ifdim\@tempb\p@<\@tempa\p@\let\@tempa\@tempb\fi% select the smaller of both
  \ifdim\@tempa\p@<\p@\scalebox{\@tempa}{\usebox\pandoc@box}%
  \else\usebox{\pandoc@box}%
  \fi%
}
% Set default figure placement to htbp
\def\fps@figure{htbp}
\makeatother
% definitions for citeproc citations
\NewDocumentCommand\citeproctext{}{}
\NewDocumentCommand\citeproc{mm}{%
  \begingroup\def\citeproctext{#2}\cite{#1}\endgroup}
\makeatletter
 % allow citations to break across lines
 \let\@cite@ofmt\@firstofone
 % avoid brackets around text for \cite:
 \def\@biblabel#1{}
 \def\@cite#1#2{{#1\if@tempswa , #2\fi}}
\makeatother
\newlength{\cslhangindent}
\setlength{\cslhangindent}{1.5em}
\newlength{\csllabelwidth}
\setlength{\csllabelwidth}{3em}
\newenvironment{CSLReferences}[2] % #1 hanging-indent, #2 entry-spacing
 {\begin{list}{}{%
  \setlength{\itemindent}{0pt}
  \setlength{\leftmargin}{0pt}
  \setlength{\parsep}{0pt}
  % turn on hanging indent if param 1 is 1
  \ifodd #1
   \setlength{\leftmargin}{\cslhangindent}
   \setlength{\itemindent}{-1\cslhangindent}
  \fi
  % set entry spacing
  \setlength{\itemsep}{#2\baselineskip}}}
 {\end{list}}
\usepackage{calc}
\newcommand{\CSLBlock}[1]{\hfill\break\parbox[t]{\linewidth}{\strut\ignorespaces#1\strut}}
\newcommand{\CSLLeftMargin}[1]{\parbox[t]{\csllabelwidth}{\strut#1\strut}}
\newcommand{\CSLRightInline}[1]{\parbox[t]{\linewidth - \csllabelwidth}{\strut#1\strut}}
\newcommand{\CSLIndent}[1]{\hspace{\cslhangindent}#1}

\usepackage{fvextra}
\DefineVerbatimEnvironment{Highlighting}{Verbatim}{breaklines,commandchars=\\\{\}}
\DefineVerbatimEnvironment{OutputCode}{Verbatim}{breaklines,commandchars=\\\{\}}
\KOMAoption{captions}{tableheading}
\makeatletter
\@ifpackageloaded{caption}{}{\usepackage{caption}}
\AtBeginDocument{%
\ifdefined\contentsname
  \renewcommand*\contentsname{Table of contents}
\else
  \newcommand\contentsname{Table of contents}
\fi
\ifdefined\listfigurename
  \renewcommand*\listfigurename{List of Figures}
\else
  \newcommand\listfigurename{List of Figures}
\fi
\ifdefined\listtablename
  \renewcommand*\listtablename{List of Tables}
\else
  \newcommand\listtablename{List of Tables}
\fi
\ifdefined\figurename
  \renewcommand*\figurename{Figure}
\else
  \newcommand\figurename{Figure}
\fi
\ifdefined\tablename
  \renewcommand*\tablename{Table}
\else
  \newcommand\tablename{Table}
\fi
}
\@ifpackageloaded{float}{}{\usepackage{float}}
\floatstyle{ruled}
\@ifundefined{c@chapter}{\newfloat{codelisting}{h}{lop}}{\newfloat{codelisting}{h}{lop}[chapter]}
\floatname{codelisting}{Listing}
\newcommand*\listoflistings{\listof{codelisting}{List of Listings}}
\makeatother
\makeatletter
\makeatother
\makeatletter
\@ifpackageloaded{caption}{}{\usepackage{caption}}
\@ifpackageloaded{subcaption}{}{\usepackage{subcaption}}
\makeatother

\usepackage{bookmark}

\IfFileExists{xurl.sty}{\usepackage{xurl}}{} % add URL line breaks if available
\urlstyle{same} % disable monospaced font for URLs
\hypersetup{
  pdftitle={PS 2},
  pdfauthor={Hyoungchul Kim},
  colorlinks=true,
  linkcolor={blue},
  filecolor={Maroon},
  citecolor={Blue},
  urlcolor={Blue},
  pdfcreator={LaTeX via pandoc}}


\title{PS 2}
\author{Hyoungchul Kim}
\date{2025-04-15}

\begin{document}
\maketitle


\section{Part 1}\label{part-1}

First, read in the data

\begin{Shaded}
\begin{Highlighting}[]
\FunctionTok{library}\NormalTok{(tidyverse)}
\FunctionTok{library}\NormalTok{(data.table)}

\NormalTok{data }\OtherTok{\textless{}{-}} \FunctionTok{read\_csv}\NormalTok{(}\StringTok{"data/middle\_kink.csv"}\NormalTok{)}

\CommentTok{\# view the data}
\NormalTok{data }\SpecialCharTok{\%\textgreater{}\%} \FunctionTok{head}\NormalTok{()}
\end{Highlighting}
\end{Shaded}

\begin{verbatim}
# A tibble: 6 x 2
  income_bin     n
       <dbl> <dbl>
1     151250   955
2     153750   982
3     156250   894
4     158750   873
5     161250   810
6     163750   851
\end{verbatim}

\subsection{a.}\label{a.}

Now we will plot the publication-quality histogram of the earnings
distribution:

\begin{Shaded}
\begin{Highlighting}[]
\FunctionTok{library}\NormalTok{(ggtext) }

\NormalTok{earning\_dist }\OtherTok{\textless{}{-}}\NormalTok{ data }\SpecialCharTok{\%\textgreater{}\%} 
  \FunctionTok{ggplot}\NormalTok{(}\FunctionTok{aes}\NormalTok{(}\AttributeTok{x=}\NormalTok{income\_bin, }\AttributeTok{y =}\NormalTok{ n)) }\SpecialCharTok{+}
  \FunctionTok{geom\_col}\NormalTok{(}\AttributeTok{fill =} \StringTok{"lightblue"}\NormalTok{) }\SpecialCharTok{+}
  \FunctionTok{geom\_point}\NormalTok{() }\SpecialCharTok{+}
  \CommentTok{\# geom\_smooth(method = "lm", se = FALSE, color = "black") +}
  \FunctionTok{geom\_line}\NormalTok{() }\SpecialCharTok{+}
  \FunctionTok{scale\_x\_continuous}\NormalTok{(}\AttributeTok{labels =}\NormalTok{ scales}\SpecialCharTok{::}\FunctionTok{label\_number}\NormalTok{(}\AttributeTok{scale =} \FloatTok{0.001}\NormalTok{, }\AttributeTok{prefix =} \StringTok{"$"}\NormalTok{)) }\SpecialCharTok{+}
  \FunctionTok{geom\_vline}\NormalTok{(}\AttributeTok{xintercept =} \DecValTok{363750}\NormalTok{, }\AttributeTok{color=}\StringTok{"red"}\NormalTok{, }\AttributeTok{shape=}\StringTok{"solid"}\NormalTok{) }\SpecialCharTok{+}
  \CommentTok{\# annotate("text", label = "kink", x = 380000, y = 800, size = 5, colour = "black") +}
  \FunctionTok{labs}\NormalTok{(}\AttributeTok{title =} \StringTok{"**Histogram of the earnings distribution**"}\NormalTok{,}
       \AttributeTok{x =} \StringTok{"**Income Bin (1000s)**"}\NormalTok{, }
       \AttributeTok{y =} \StringTok{"**Number of Observations**"}\NormalTok{) }\SpecialCharTok{+}
  \FunctionTok{theme\_bw}\NormalTok{() }\SpecialCharTok{+} 
  \FunctionTok{theme}\NormalTok{(}
    \AttributeTok{plot.title =} \FunctionTok{element\_markdown}\NormalTok{(}\AttributeTok{size =} \DecValTok{16}\NormalTok{, }\AttributeTok{hjust =} \FloatTok{0.5}\NormalTok{),}
    \AttributeTok{axis.title.x =} \FunctionTok{element\_markdown}\NormalTok{(}\AttributeTok{size =} \DecValTok{14}\NormalTok{),}
    \AttributeTok{axis.title.y =} \FunctionTok{element\_markdown}\NormalTok{(}\AttributeTok{size =} \DecValTok{14}\NormalTok{),}
    \AttributeTok{axis.text.x =} \FunctionTok{element\_text}\NormalTok{(}\AttributeTok{size =} \DecValTok{12}\NormalTok{),}
    \AttributeTok{axis.text.y =} \FunctionTok{element\_text}\NormalTok{(}\AttributeTok{size =} \DecValTok{12}\NormalTok{)}
\NormalTok{  )}
\NormalTok{earning\_dist}
\end{Highlighting}
\end{Shaded}

\pandocbounded{\includegraphics[keepaspectratio]{report_files/figure-pdf/unnamed-chunk-2-1.pdf}}

\subsection{b.}\label{b.}

We will be following Saez (2010) to construct the equation to retrieve
the elasticity \(e\). Note for our case, the kink happens at
\(z^* = 365000\) and marginal tax rate changes from \(0.07\) to
\(0.21\). We need to use equation (5) in the paper to get the
elasticity. The equation is as follows:

\[
  B = z^* \left[ \left( \frac{1-t_0}{1-t_1} \right)^e - 1 \right] \frac{h(z^*)\_ + h(z^*)_+ \bigg/ \left( \frac{1-t_0}{1-t_1} \right)^e}{2}.
\]

In order to compute \(B\), we need to decide \(\delta\) to calculate the
width we will use to calculate excess bunching. We will use the
``simplest method'' mentioned in the paper which is to select \(\delta\)
graphically such that the full excess bunching is included in the band
\((z^* - \delta + z^* + \delta)\). In our case, it seems to be about
\(\delta = 4\) (Note that since our data is in income bin of width
2,500, this is equivalent to 10,000 difference). Numerically, it will be
calculated as follows (this is just following the equation (6) in the
paper):

\begin{Shaded}
\begin{Highlighting}[]
\NormalTok{A }\OtherTok{\textless{}{-}}\NormalTok{ data }\SpecialCharTok{\%\textgreater{}\%} \FunctionTok{count}\NormalTok{(}\AttributeTok{wt=}\NormalTok{n) }\SpecialCharTok{\%\textgreater{}\%} \FunctionTok{pull}\NormalTok{()}

\NormalTok{b }\OtherTok{\textless{}{-}} \ControlFlowTok{function}\NormalTok{(delta) \{}

\NormalTok{B1  }\OtherTok{\textless{}{-}}\NormalTok{ data }\SpecialCharTok{\%\textgreater{}\%} 
  \FunctionTok{filter}\NormalTok{(income\_bin }\SpecialCharTok{\textgreater{}=} \DecValTok{365000} \SpecialCharTok{{-}} \DecValTok{2500}\SpecialCharTok{*}\NormalTok{delta }\SpecialCharTok{\&}\NormalTok{ income\_bin }\SpecialCharTok{\textless{}=} \DecValTok{365000} \SpecialCharTok{+} \DecValTok{2500}\SpecialCharTok{*}\NormalTok{delta) }\SpecialCharTok{\%\textgreater{}\%}  \CommentTok{\# 25,000 differences}
  \FunctionTok{count}\NormalTok{(}\AttributeTok{wt=}\NormalTok{n) }\SpecialCharTok{\%\textgreater{}\%} 
  \FunctionTok{pull}\NormalTok{() }\CommentTok{\#8,574}

\NormalTok{B2 }\OtherTok{\textless{}{-}}\NormalTok{ data }\SpecialCharTok{\%\textgreater{}\%} 
  \FunctionTok{filter}\NormalTok{(income\_bin }\SpecialCharTok{\textgreater{}=} \DecValTok{365000} \SpecialCharTok{{-}} \DecValTok{2500}\SpecialCharTok{*}\DecValTok{2}\SpecialCharTok{*}\NormalTok{delta }\SpecialCharTok{\&}\NormalTok{ income\_bin }\SpecialCharTok{\textless{}=} \DecValTok{365000} \SpecialCharTok{{-}} \DecValTok{2500}\SpecialCharTok{*}\NormalTok{delta) }\SpecialCharTok{\%\textgreater{}\%} 
  \FunctionTok{count}\NormalTok{(}\AttributeTok{wt=}\NormalTok{n) }\SpecialCharTok{\%\textgreater{}\%} 
  \FunctionTok{pull}\NormalTok{() }\CommentTok{\# 3615}

\NormalTok{B3 }\OtherTok{\textless{}{-}}\NormalTok{ data }\SpecialCharTok{\%\textgreater{}\%} 
  \FunctionTok{filter}\NormalTok{(income\_bin }\SpecialCharTok{\textgreater{}=} \DecValTok{365000} \SpecialCharTok{+} \DecValTok{2500}\SpecialCharTok{*}\NormalTok{delta }\SpecialCharTok{\&}\NormalTok{ income\_bin }\SpecialCharTok{\textless{}=} \DecValTok{365000} \SpecialCharTok{+} \DecValTok{2500}\SpecialCharTok{*}\DecValTok{2}\SpecialCharTok{*}\NormalTok{delta) }\SpecialCharTok{\%\textgreater{}\%} 
  \FunctionTok{count}\NormalTok{(}\AttributeTok{wt=}\NormalTok{n) }\SpecialCharTok{\%\textgreater{}\%} 
  \FunctionTok{pull}\NormalTok{() }\CommentTok{\# 2982}

\NormalTok{B }\OtherTok{=}\NormalTok{ (B1 }\SpecialCharTok{{-}}\NormalTok{ B2 }\SpecialCharTok{{-}}\NormalTok{ B3) }\SpecialCharTok{/}\NormalTok{ A}

\NormalTok{\}}

\NormalTok{b1 }\OtherTok{\textless{}{-}} \ControlFlowTok{function}\NormalTok{(delta) \{}

\NormalTok{B1  }\OtherTok{\textless{}{-}}\NormalTok{ data }\SpecialCharTok{\%\textgreater{}\%} 
  \FunctionTok{filter}\NormalTok{(income\_bin }\SpecialCharTok{\textgreater{}=} \DecValTok{365000} \SpecialCharTok{{-}} \DecValTok{2500}\SpecialCharTok{*}\NormalTok{delta }\SpecialCharTok{\&}\NormalTok{ income\_bin }\SpecialCharTok{\textless{}=} \DecValTok{365000} \SpecialCharTok{+} \DecValTok{2500}\SpecialCharTok{*}\NormalTok{delta) }\SpecialCharTok{\%\textgreater{}\%}  \CommentTok{\# 25,000 differences}
  \FunctionTok{count}\NormalTok{(}\AttributeTok{wt=}\NormalTok{n) }\SpecialCharTok{\%\textgreater{}\%} 
  \FunctionTok{pull}\NormalTok{() }\CommentTok{\#8,574}

\NormalTok{\}}

\NormalTok{b2 }\OtherTok{\textless{}{-}} \ControlFlowTok{function}\NormalTok{(delta) \{}

\NormalTok{B2 }\OtherTok{\textless{}{-}}\NormalTok{ data }\SpecialCharTok{\%\textgreater{}\%} 
  \FunctionTok{filter}\NormalTok{(income\_bin }\SpecialCharTok{\textgreater{}=} \DecValTok{365000} \SpecialCharTok{{-}} \DecValTok{2500}\SpecialCharTok{*}\DecValTok{2}\SpecialCharTok{*}\NormalTok{delta }\SpecialCharTok{\&}\NormalTok{ income\_bin }\SpecialCharTok{\textless{}=} \DecValTok{365000} \SpecialCharTok{{-}} \DecValTok{2500}\SpecialCharTok{*}\NormalTok{delta) }\SpecialCharTok{\%\textgreater{}\%} 
  \FunctionTok{count}\NormalTok{(}\AttributeTok{wt=}\NormalTok{n) }\SpecialCharTok{\%\textgreater{}\%} 
  \FunctionTok{pull}\NormalTok{() }\CommentTok{\# 3615}

\NormalTok{\}}

\NormalTok{b3 }\OtherTok{\textless{}{-}} \ControlFlowTok{function}\NormalTok{(delta) \{}

\NormalTok{B3 }\OtherTok{\textless{}{-}}\NormalTok{ data }\SpecialCharTok{\%\textgreater{}\%} 
  \FunctionTok{filter}\NormalTok{(income\_bin }\SpecialCharTok{\textgreater{}=} \DecValTok{365000} \SpecialCharTok{+} \DecValTok{2500}\SpecialCharTok{*}\NormalTok{delta }\SpecialCharTok{\&}\NormalTok{ income\_bin }\SpecialCharTok{\textless{}=} \DecValTok{365000} \SpecialCharTok{+} \DecValTok{2500}\SpecialCharTok{*}\DecValTok{2}\SpecialCharTok{*}\NormalTok{delta) }\SpecialCharTok{\%\textgreater{}\%} 
  \FunctionTok{count}\NormalTok{(}\AttributeTok{wt=}\NormalTok{n) }\SpecialCharTok{\%\textgreater{}\%} 
  \FunctionTok{pull}\NormalTok{() }\CommentTok{\# 2982}
\NormalTok{\}}

\NormalTok{B }\OtherTok{\textless{}{-}} \FunctionTok{b}\NormalTok{(}\DecValTok{4}\NormalTok{)}
\NormalTok{B1 }\OtherTok{\textless{}{-}} \FunctionTok{b1}\NormalTok{(}\DecValTok{4}\NormalTok{)}
\NormalTok{B2 }\OtherTok{\textless{}{-}} \FunctionTok{b2}\NormalTok{(}\DecValTok{4}\NormalTok{)}
\NormalTok{B3 }\OtherTok{\textless{}{-}} \FunctionTok{b3}\NormalTok{(}\DecValTok{4}\NormalTok{)}
\end{Highlighting}
\end{Shaded}

Note that we computed \(B\) in the equation by getting the total
fraction of people that are in the excess bunching.

Now we also need to compute two \(h\) in the main equation. Empirically
we can calculate this by dividing B2, B3 by \(\delta\) respectively.
They are the left and right density point around the bunching.

\begin{Shaded}
\begin{Highlighting}[]
\NormalTok{h\_min }\OtherTok{=}\NormalTok{ (B2 }\SpecialCharTok{/} \DecValTok{5000}\NormalTok{) }\SpecialCharTok{/}\NormalTok{ A}
\NormalTok{h\_plus }\OtherTok{=}\NormalTok{ (B3 }\SpecialCharTok{/} \DecValTok{5000}\NormalTok{) }\SpecialCharTok{/}\NormalTok{ A }

\NormalTok{h\_min}
\end{Highlighting}
\end{Shaded}

\begin{verbatim}
[1] 5.891811e-06
\end{verbatim}

\begin{Shaded}
\begin{Highlighting}[]
\NormalTok{h\_plus}
\end{Highlighting}
\end{Shaded}

\begin{verbatim}
[1] 4.648559e-06
\end{verbatim}

Finally, we can plug in the values we got from the data and get the
elasticity \(e\). Here, we are just basically getting the solution by
plugging in the empirical numbers we computed from the data into the
main equation:

\begin{Shaded}
\begin{Highlighting}[]
\CommentTok{\# Define the function whose root we want to find}
\NormalTok{f }\OtherTok{\textless{}{-}} \ControlFlowTok{function}\NormalTok{(e) \{}

\NormalTok{  z  }\OtherTok{\textless{}{-}} \DecValTok{365000}
\NormalTok{  ratio }\OtherTok{\textless{}{-}}\NormalTok{ (}\DecValTok{1} \SpecialCharTok{{-}} \FloatTok{0.07}\NormalTok{) }\SpecialCharTok{/}\NormalTok{ (}\DecValTok{1} \SpecialCharTok{{-}} \FloatTok{0.21}\NormalTok{)}

\NormalTok{  z }\SpecialCharTok{*}\NormalTok{ (ratio}\SpecialCharTok{\^{}}\NormalTok{e }\SpecialCharTok{{-}} \DecValTok{1}\NormalTok{) }\SpecialCharTok{*}\NormalTok{ ( (h\_min }\SpecialCharTok{+}\NormalTok{ (h\_plus }\SpecialCharTok{/}\NormalTok{ ratio}\SpecialCharTok{\^{}}\NormalTok{e)) }\SpecialCharTok{/} \DecValTok{2}\NormalTok{) }\SpecialCharTok{{-}}\NormalTok{ B}
\NormalTok{\}}

\CommentTok{\# Use uniroot to solve f(e) = 0 in a reasonable range for e}
\NormalTok{result }\OtherTok{\textless{}{-}} \FunctionTok{uniroot}\NormalTok{(f, }\AttributeTok{lower =} \SpecialCharTok{{-}}\DecValTok{1}\NormalTok{, }\AttributeTok{upper =} \DecValTok{1}\NormalTok{)}

\CommentTok{\# Extract the solution}
\NormalTok{e\_solution }\OtherTok{\textless{}{-}}\NormalTok{ result}\SpecialCharTok{$}\NormalTok{root}
\FunctionTok{print}\NormalTok{(e\_solution)}
\end{Highlighting}
\end{Shaded}

\begin{verbatim}
[1] 0.05340052
\end{verbatim}

\subsection{c.}\label{c.}

Using \texttt{bunchr} package in R, I plot the counterfactual density
and get the estimate of the elasticity as follows:

\begin{Shaded}
\begin{Highlighting}[]
\FunctionTok{library}\NormalTok{(bunchr)}
\CommentTok{\# \# analyzing a kink}
\CommentTok{\# ability\_vec \textless{}{-} 4000 * rbeta(100000, 2, 5)}
\CommentTok{\# earning\_vec \textless{}{-} sapply(ability\_vec, earning\_fun, 0.2, 0, 0.2, 0, 1000)}
\CommentTok{\# earning\_vec}
\CommentTok{\# \# bunch\_viewer(earning\_vec, 1000, 20, 20, 1, 1, binw = 20)}
\CommentTok{\# estim \textless{}{-} bunch(earning\_vec, 1000, 0, 0.2, Tax = 0, 20, 20, 1, 1,}
\CommentTok{\# binw = 20, draw=TRUE, nboots = 0, seed = 16)}
\CommentTok{\# estim$e}

\CommentTok{\# Step 1: Expand binned data into a raw vector}
\NormalTok{z\_vector }\OtherTok{\textless{}{-}}\NormalTok{ data }\SpecialCharTok{\%\textgreater{}\%}
  \FunctionTok{rowwise}\NormalTok{() }\SpecialCharTok{\%\textgreater{}\%}
  \FunctionTok{summarise}\NormalTok{(}\AttributeTok{vec =} \FunctionTok{list}\NormalTok{(}\FunctionTok{rep}\NormalTok{(income\_bin, n))) }\SpecialCharTok{\%\textgreater{}\%}
  \FunctionTok{pull}\NormalTok{(vec) }\SpecialCharTok{\%\textgreater{}\%}
  \FunctionTok{unlist}\NormalTok{()}

\CommentTok{\# Step 2: Estimate bunching}
\NormalTok{estim }\OtherTok{\textless{}{-}} \FunctionTok{bunch}\NormalTok{(}
  \AttributeTok{earnings =}\NormalTok{ z\_vector,}
  \AttributeTok{zstar =} \DecValTok{365000}\NormalTok{,}
  \AttributeTok{binw =} \DecValTok{2500}\NormalTok{,}
  \AttributeTok{t1 =} \FloatTok{0.07}\NormalTok{,}
  \AttributeTok{t2 =} \FloatTok{0.21}\NormalTok{,}
  \AttributeTok{cf\_start =} \DecValTok{8}\NormalTok{,}
  \AttributeTok{cf\_end =} \DecValTok{10}\NormalTok{,}
  \AttributeTok{exclude\_before =} \DecValTok{4}\NormalTok{,}
  \AttributeTok{exclude\_after =} \DecValTok{3}\NormalTok{,}
  \CommentTok{\# poly\_size = 2,}
  \AttributeTok{draw =} \ConstantTok{FALSE}\NormalTok{,}
  \AttributeTok{nboots =} \DecValTok{100}
\NormalTok{)}
\end{Highlighting}
\end{Shaded}

\pandocbounded{\includegraphics[keepaspectratio]{bunching.png}}

The elasticity estimate is as follows:

\begin{Shaded}
\begin{Highlighting}[]
\CommentTok{\# Estimate of the elasticity from the package}
\NormalTok{estim}\SpecialCharTok{$}\NormalTok{e}
\end{Highlighting}
\end{Shaded}

\begin{verbatim}
[1] 0.1189242
\end{verbatim}

\subsection{d.}\label{d.}

The elasticity in (b) does not match the one in (c). It may be due to
several reasons. First, in (b) I set \(\delta=4\) which is different
from (c) and I also set asymmetric bandwidth in \texttt{bunchr} package.
Due to these different settings, the results would be different from
each other. While both results are bit different from the working paper
by Anagol et al. (2024), the result I got from (c) seems to be more
close to the working paper. This is likely because, like Chetty et al.
(2011), Anagol et al. (2024) proposes a method for estimating \(B\)
using the observed density, without requiring full knowledge of the
counterfactual densities. However, Anagol et al. (2024)'s approach
differs somewhat from that of Chetty et al. (2011) and other related
work. While Chetty et al. (2011) estimates \(B\) by integrating over the
observed density relative to a counterfactual density estimated via
polynomial fitting, Anagol et al. (2024) takes a more explicit approach
to modeling the diffusion process---separately identifying the diffuse
bunching expected near a kink from the underlying counterfactual
distribution.

\section{Part 2}\label{part-2}

There is no straightforward answer to this question. Before we started
to consider the dynamic nature of the taxation policy, the consensus was
that we should not tax the capital. The reason was similar to our
consensus on not wanting to tax the commodity. This is due to two main
classical answers to this question. First, Chamley-Judd results told us
that we should not tax capital in the long run.\footnote{But note that
  Chamley-Judd results were recently overturned in Straub and Werning
  (AER, 2020) so we should be cautious about their results.} Secondly,
Atkinson-Stiglitz results told us that we should only be taxing through
income as taxing through commodity or capital will further distort the
behavior of people.

But our consensus could change if we move onto a dynamic setting. In
fact, some papers have shown that the capital taxation should be
positive in the dynamic setting. For example, savings (capital) wedge
generally becomes positive in dynamic mechanism design solution for
mirlees problem. The intuition for this is because in the dynamic
setting people try to save more for the future to account for any
possible risk in the future such as negative shock on their
ability.Hence it leads to a negative fiscal externality for the
government revenue. In order to alleviate this, the government should
try to tax the capital. In this case, we could say it should tax the
capital to offset the externality. However, setting positive capital
taxation might not be that important in terms of policy perspective.
This is because some papers (e.g.~Farhi and Werning 2013) have shown
that while it is true that capital tax is positive in dynamic setting,
its actual magnitude seems to be small (close to zero). Thus in real
life policy, it might not make much of a difference whether you tax
capital (very slightly) or not.

Furthermore, there are some other methodological approaches to think
about optimal capital taxation. For instance, there are some papers such
as dynamic Ramsey setup where they impose certain parametric form for
the tax function. In this case, optimal tax can be positive depending on
the tax policy instrument restrictions. Also, capital taxation can be
positive in sufficient statistics approach. Still, whether capital
taxation has to be indeed positive will again depend on certain
conditions that have no definitive answer.\footnote{For example, we
  would need to consider whether capital taxation is just a workaround
  for some other features of dynamic constrained optimum for dynamic
  Ramsey. For sufficient statistics case, we would need to have a
  structural model for full calibration.}

\subsection*{References}\label{references}
\addcontentsline{toc}{subsection}{References}

\phantomsection\label{refs}
\begin{CSLReferences}{1}{0}
\bibitem[\citeproctext]{ref-anagol}
Anagol, Santosh, Allan Davids, Benjamin B Lockwood, and Tarun Ramadorai.
2024. {``Diffuse Bunching with Frictions: Theory and Estimation.''}
Working Paper 32597. Working Paper Series. National Bureau of Economic
Research. \url{https://doi.org/10.3386/w32597}.

\bibitem[\citeproctext]{ref-chetty}
Chetty, Raj, John N. Friedman, Tore Olsen, and Luigi Pistaferri. 2011.
{``Adjustment Costs, Firm Responses, and Micro Vs. Macro Labor Supply
Elasticities: Evidence from Danish Tax Records *.''} \emph{The Quarterly
Journal of Economics} 126 (2): 749--804.
\url{https://doi.org/10.1093/qje/qjr013}.

\bibitem[\citeproctext]{ref-saez2010}
Saez, Emmanuel. 2010. {``Do Taxpayers Bunch at Kink Points?''}
\emph{American Economic Journal: Economic Policy} 2 (3): 180--212.
\url{https://doi.org/10.1257/pol.2.3.180}.

\end{CSLReferences}




\end{document}
